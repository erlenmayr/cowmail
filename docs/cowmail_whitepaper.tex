\documentclass[a4paper]{article}

\usepackage{mathpazo}
\usepackage[T1]{fontenc}
\usepackage[utf8]{inputenc}

\usepackage{amsmath}
\usepackage{amsfonts}
\usepackage{amssymb}
\usepackage{amsthm}



\title{Cowmail}
\author{Stephan Verbücheln}
\date{November 23, 2020}

\begin{document}

\maketitle


\begin{abstract}
Cowmail is a new offline messaging system using a novel addressing mechanism.
Its goal is to provide the following features:
\begin{itemize}
\item End-to-end encryption
\item Sender anonymity
\item Recipient anonymity
\end{itemize}
\end{abstract}

\section{Introduction}
TODO

\section{Cryptography}
Messages are encrypted with the recipients elliptic-curve public key. The
encryption is based on the GnuTLS implementations of AES-GCM and Curve25519 in
an ElGamal-style scheme.

\subsection{Key Generation}
The recipient Bob has a Curve25519 key pair $(b, B)$, of which the sender Alice
knows the public part $B$. For each message, Alice generates a new key pair
$(a, A)$. From both keys combined, a temporary AES key $k$ is derived.
\begin{align*}
a &= RNG(256) \\
A &= a G \\
k &= a B = b A \\
\end{align*}

\subsection{Payload Encryption}
For a clear text message $msg$, the resulting encrypted payload is called
message $body$. The encryption is defined as follows based on the AES-GCM
cryptotext and authentication tag.
\begin{align*}
iv    &= A[16..31] \\
ctext &= encrypt(iv, k, msg) \\
mtag  &= authenticate(iv, k, msg) \\
body  &= mtag || ctext
\end{align*}

\subsection{Header Encryption}
For addressing purposes, a message $head$ is generated for each message. The
head contains the per-message Curve25519 public key as well as an encryped hash
of the payload.
\begin{align*}
iv    &= A[0..15]\\
hash  &= SHA256(body) \\
chash &= encrypt(iv, k, msg) \\
ctag  &= authenticate(iv, k, msg) \\
head  &= A || chash || ctag
\end{align*}

\subsection{Binary Message format}
The resulting binary message format looks as follows:
\begin{center}
\begin{tabular}{ |ccc|cc| }
 \hline
 head &&& body & \\
 (96 bytes) &&& (variable length) & \\
 \hline
 A & chash & ctag & mtag & ctext \\
 (32 byes) & (32 bytes) & (16 bytes) & (16 bytes) & (variable length) \\
 \hline
\end{tabular}
\end{center}

\section{Network Protocol}

\subsection{Commands}
The protocol requests are all text-based. Message \textit{body} and
\textit{head} are always encoded with \textit{base64}. Hashes are transferred as
hexadecimal string. At the moment, there are only three commands.

\textbf{LIST} Lists the message \textit{heads} of all messages that are
available on the server. The command has no arguments.
\begin{verbatim}
LIST
\end{verbatim}
It returns all message \textit{heads}.

\textbf{GET} receives the message \textit{body} for a given message
\textit{hash}. The message \textit{hash} is the only argument.
\begin{verbatim}
GET {hash}
\end{verbatim}
It returns the message for the \textit{hash}.

\textbf{PUT} allows to put a message. For this, two arguments are required:
Message \textit{head} and message \textit{body}.
\begin{verbatim}
PUT {head} {body}
\end{verbatim}
No data is returned.

\subsection{SCTP}
As an experimental project, “new” technologies, Cowmail's network protocol is
based on SCTP and IPv6. SCTP is perfect for an offline messaging system. IPv6
provides a nice basis for a decentralized network where everyone can operate
his own server. However, in case there are problems with using SCTP in
production (e.g. due to lack of support in firewalls), Cowmail could easily be
ported to TCP or UDP.

\section{Side Channels}
One has to be aware that there might be a number of side channels jeopardizing
the anonymity. For instance, IP addresses might reveal the identity.

\section{Roadmap}
For the future, the following features are planned:
\begin{itemize}
\item Proof-of-work for spam prevention
\item Server keys
\item Transport encryption
\item Address database
\end{itemize}

\begin{thebibliography}{ABCDE99}

\bibitem[BDLSY12]{eddsa} 
Daniel J. Bernstein, Niels Duif, Tanja Lange, Peter Schwabe, Bo-Yin Yang, 
\textit{High-speed high-security signatures}, 
Journal of Cryptographic Engineering,
September 2012, Volume 2, Issue 2, pp 77-89,
2012, Springer-Verlag

\end{thebibliography}

\end{document}
