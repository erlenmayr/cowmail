\documentclass[a4paper]{article}

\usepackage{mathpazo}
\usepackage[T1]{fontenc}
\usepackage[utf8]{inputenc}

\usepackage{amsmath}
\usepackage{amsfonts}
\usepackage{amssymb}
\usepackage{amsthm}

\usepackage{tikz}
\usetikzlibrary{positioning}



\title{Cowmail}
\author{Stephan Verbücheln}
\date{October 9, 2020}

\begin{document}

\maketitle


\begin{abstract}
Cowmail is a new offline messaging system using a novel addressing mechanism. Its
goal is to provide the following features:
\begin{itemize}
\item End-to-end encryption
\item Sender anonymity
\item Recipient anonymity
\end{itemize}
\end{abstract}

\section{Introduction}
TODO

\section{Cryptography}
\subsection{Payload encryption}
Messages are encrypted with the recipients elliptic-curve public key. The
encryption is based on the \textit{sealed box}, an ElGamal-based scheme provided
by \textit{libsodium}. It is based on Curve25519\cite{eddsa}.

The result is called message \textit{body}, the encryption works as follows:

$$
body = sealed\_box(message, pkey)
$$

\subsection{Addressing mechanism}
For addressing purposes, a message \textit{head} is generated for each message.
The message \textit{hash} is based on a hash of the encrypted message \textit{body}.

$$
hash = SHA256(body)
$$
$$
head = sealed\_box(hash, pkey)
$$

\section{Protocol}

\subsection{Commands}
The protocol requests are all text-based. Message \textit{body} and
\textit{head} are always encoded with \textit{base64}. Hashes are transferred as
hexadecimal string. At the moment, there are only three commands.

\textbf{LIST} Lists the message \textit{heads} of all messages that are
available on the server. The command has no arguments.
\begin{verbatim}
LIST
\end{verbatim}
It returns all message \textit{heads}.

\textbf{GET} receives the message \textit{body} for a given message
\textit{hash}. The message \textit{hash} is the only argument.
\begin{verbatim}
GET {hash}
\end{verbatim}
It returns the message for the \textit{hash}.

\textbf{PUT} allows to put a message. For this, two arguments are required:
Message \textit{head} and message \textit{body}.
\begin{verbatim}
PUT {head} {body}
\end{verbatim}
No data is returned.

\subsection{SCTP}
As an experimental project, “new” technologies, the Cowmail implementation is
based on SCTP and IPv6. SCTP is perfect for an offline messaging system. IPv6
provides a nice basis for a decentralized network where everyone can operate
his own server. However, in case there are problems with using SCTP in
production (e.g. due to lack of support in Firewalls), Cowmail could easily be
ported to TCP or UDP.

\section{Side Channels}
One has to be aware that there might be a number of side channels jeopardizing
the anonymity. For instance, IP addresses might reveal the identity.

\section{Roadmap}
For the future, the following features are planned:
\begin{itemize}
\item Proof-of-work for spam prevention
\item Server keys
\item Transport encryption
\item Address database
\end{itemize}

\begin{thebibliography}{ABCDE99}

\bibitem[BDLSY12]{eddsa} 
Daniel J. Bernstein, Niels Duif, Tanja Lange, Peter Schwabe, Bo-Yin Yang, 
\textit{High-speed high-security signatures}, 
Journal of Cryptographic Engineering,
September 2012, Volume 2, Issue 2, pp 77-89,
2012, Springer-Verlag

\end{thebibliography}

\end{document}

